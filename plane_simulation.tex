\documentclass{article}
\usepackage{amsmath}
\usepackage{algorithm}
\usepackage{algorithmic}
\usepackage{graphicx}

\title{3D Flight Simulator: Implementation of \\
Computer Graphics Concepts}
\author{Computer Graphics Project Team\\\small{Waseda University}}
\date{January 2025}

\begin{document}

\maketitle

\begin{abstract}
This paper provides a detailed technical analysis of a 3D flight simulator implemented using Three.js, focusing on the mathematical foundations and computer graphics concepts utilized. The implementation demonstrates advanced graphics techniques including quaternion-based rotations, physics-based animations, particle systems, procedural terrain generation, real-time rendering optimizations, minimap integration, HUD systems, and custom shaders.
\end{abstract}

\section{Mathematical Foundations}

\subsection{3D Transformations}
The core transformation pipeline uses homogeneous coordinates and matrix operations:

\subsubsection{Rotation Matrices}
Euler angle rotations are implemented using:
\begin{equation}
R_x(\theta) = \begin{pmatrix}
1 & 0 & 0 & 0 \\
0 & \cos\theta & -\sin\theta & 0 \\
0 & \sin\theta & \cos\theta & 0 \\
0 & 0 & 0 & 1
\end{pmatrix}
\end{equation}

\subsubsection{Quaternion Rotations}
Aircraft orientation uses quaternions to avoid gimbal lock:
\begin{equation}
q = \cos\left(\frac{\theta}{2}\right) + (x\mathbf{i} + y\mathbf{j} + z\mathbf{k})\sin\left(\frac{\theta}{2}\right)
\end{equation}

Quaternion multiplication for rotation composition:
\begin{equation}
q_1 \otimes q_2 = (w_1w_2 - \vec{v_1}\cdot\vec{v_2}, w_1\vec{v_2} + w_2\vec{v_1} + \vec{v_1}\times\vec{v_2})
\end{equation}

\section{Flight Physics System}

\subsection{Aerodynamics Model}
The flight model implements the following forces:

\subsubsection{Lift Force}
\begin{equation}
F_{lift} = \frac{1}{2}\rho v^2 S C_L \sin(\alpha)
\end{equation}
where:
\begin{itemize}
\item $\rho$ = air density (1.225 kg/m\textsuperscript{3})
\item $v$ = airspeed
\item $S$ = wing area
\item $C_L$ = lift coefficient
\item $\alpha$ = angle of attack
\end{itemize}

\subsubsection{Drag Force}
\begin{equation}
F_{drag} = \frac{1}{2}\rho v^2 S C_D
\end{equation}
where $C_D = C_{D0} + kC_L^2$ (parasitic and induced drag)

\section{Particle Systems}
Particle motion follows Newtonian physics:

\subsection{Position Update}
\begin{equation}
\vec{p}(t) = \vec{p}_0 + \vec{v}_0t + \frac{1}{2}\vec{a}t^2
\end{equation}

\subsection{Particle Life Cycle}
Opacity calculation:
\begin{equation}
opacity = \frac{life_{remaining}}{life_{total}}
\end{equation}

\section{Terrain Generation}
Multi-octave noise function:
\begin{equation}
height(x,z) = \sum_{i=1}^n A_i\sin(f_ix)\cos(f_iz)
\end{equation}
where:
\begin{itemize}
\item $A_i$ = amplitude for octave $i$
\item $f_i$ = frequency for octave $i$
\end{itemize}

\section{HUD System}
The HUD displays key metrics such as speed, altitude, pitch, roll, G-force, and fuel percentage. These values are computed using:
\begin{itemize}
\item Speed ($v$):
\begin{equation}
v = \sqrt{v_x^2 + v_y^2 + v_z^2}
\end{equation}
\item Altitude ($h$):
\begin{equation}
h = p_y
\end{equation}
\item Pitch ($\theta_{pitch}$):
\begin{equation}
\theta_{pitch} = \arcsin\left(\frac{v_y}{v}\right)
\end{equation}
\item Roll ($\theta_{roll}$):
\begin{equation}
\theta_{roll} = \arctan2(v_x, v_z)
\end{equation}
\item G-force ($g$):
\begin{equation}
g = 1 + \frac{|\vec{a}|}{g_0}
\end{equation}
where $g_0$ is the acceleration due to gravity.
\end{itemize}

\section{Minimap System}
The minimap provides a top-down view of the airplane's position and heading. The coordinates are scaled as follows:
\begin{equation}
x_{map} = \frac{p_x}{scale} + center_x, \quad z_{map} = \frac{p_z}{scale} + center_z
\end{equation}

The direction vector is computed using:
\begin{equation}
\vec{d} = R(\vec{d}_{initial})
\end{equation}
where $R$ is the rotation matrix based on the airplane's current orientation.

\section{Missile System}
Missiles follow a linear trajectory with particle-based trail effects. The missile's position updates are computed using:
\begin{equation}
\vec{p}_{missile}(t) = \vec{p}_{launch} + \vec{v}_{missile}t
\end{equation}

Trail particles decay over time, with their opacity defined as:
\begin{equation}
opacity_{trail} = \frac{life_{remaining}}{life_{total}}
\end{equation}

\section{Shaders and Rendering Techniques}
Shaders play a critical role in enhancing the visual quality of the simulator. The following shaders are implemented:

\subsection{Sky Gradient Shader}
A custom shader is used to create a gradient sky. The vertex and fragment shaders are defined as:
\begin{itemize}
\item Vertex Shader:
\begin{equation}
v_{position} = modelMatrix \cdot vec4(position, 1.0)
\end{equation}
\item Fragment Shader:
\begin{equation}
color = mix(bottomColor, topColor, max(normalize(v_{worldPosition}).y, 0.0))
\end{equation}
\end{itemize}

\subsection{Phong Lighting}
Implemented in shaders for dynamic lighting effects. The lighting equation is:
\begin{equation}
I = k_a i_a + k_d i_d(\vec{L}\cdot\vec{N}) + k_s i_s(\vec{R}\cdot\vec{V})^n
\end{equation}

\section{Graphics Pipeline Implementation}
\subsection{Render Loop Algorithm}
\begin{algorithmic}[1]
\STATE Update physics state
\begin{equation}
\vec{F}_{total} = \vec{F}_{thrust} + \vec{F}_{lift} + \vec{F}_{drag} + \vec{F}_{gravity}
\end{equation}
\STATE Update velocities
\begin{equation}
\vec{v}_{new} = \vec{v}_{old} + \frac{\vec{F}_{total}}{m}\Delta t
\end{equation}
\STATE Update positions
\begin{equation}
\vec{p}_{new} = \vec{p}_{old} + \vec{v}_{new}\Delta t
\end{equation}
\STATE Update rotations via quaternion
\begin{equation}
q_{new} = q_{old} \otimes \Delta q
\end{equation}
\STATE Update particle systems
\STATE Perform collision detection
\STATE Update camera matrix
\begin{equation}
M_{view} = M_{translation}M_{rotation}
\end{equation}
\STATE Render scene
\end{algorithmic}

\section{Lighting and Materials}

\subsection{Phong Lighting Model}
Implemented using:
\begin{equation}
I = k_a i_a + k_d i_d(\vec{L}\cdot\vec{N}) + k_s i_s(\vec{R}\cdot\vec{V})^n
\end{equation}
where:
\begin{itemize}
\item $k_a$, $k_d$, $k_s$ = material coefficients
\item $i_a$, $i_d$, $i_s$ = light intensities
\item $\vec{L}$, $\vec{N}$, $\vec{R}$, $\vec{V}$ = light, normal, reflection, and view vectors
\end{itemize}

\section{Camera System}
Camera follows aircraft using smooth interpolation:
\begin{equation}
p_{camera} = lerp(p_{current}, p_{target}, \alpha)
\end{equation}
where $\alpha = 0.1$ for smooth transition.

\section{Collision Detection}
Ground collision implemented using height-based detection:
\begin{equation}
collision = p_y < terrain_{height}(p_x, p_z) + offset
\end{equation}

\section{Performance Optimizations}

\subsection{LOD System}
Level of detail calculation:
\begin{equation}
detail_{level} = \left\lfloor\log_2\left(\frac{distance}{base_{distance}}\right)\right\rfloor
\end{equation}

\section{Results}
The implementation achieves:
\begin{itemize}
\item 60+ FPS rendering performance
\item Realistic flight physics
\item Particle system with 1000+ active particles
\item Dynamic terrain generation
\item Smooth camera transitions
\item Real-time collision detection
\item HUD and minimap integration
\item Custom shader effects for sky and lighting
\end{itemize}

\section{Conclusion}
This implementation demonstrates the practical application of advanced computer graphics concepts, combining mathematical principles with real-time rendering techniques to create an interactive flight simulation environment.

\end{document}
